\documentclass[12pt, a4paper,]{article}

\usepackage[utf8x]{inputenc}
\usepackage[english,russian]{babel}
\usepackage{multicol}
\usepackage{amsmath}
\usepackage{amssymb}
\usepackage{fancyhdr}
\usepackage{graphicx}
\usepackage[left=20mm, bottom=18mm, top=30mm]{geometry}
\textwidth=17cm
\pagestyle{fancy}
\fancyhead{}
\fancyhead[L]{\normalsize \textsc{\thepage}}
\fancyhead[C]{\textbf{\footnotesize К В А Н Т }$\cdot$ \textsc{2 0 1 7 / № 3}}
\fancyfoot{}
\renewcommand{\headrulewidth}{0pt}
\setcounter{page}{36}
\begin{document}%
\noindent
\begin{minipage}[top]{.55\textwidth}

\footnotesize
\resizebox{\linewidth}{!}{%
\begin{tabular}{|c|c|c|c|c|c|c|c|c|}

\hline
$\mathrm{Xe}$ & $\mathrm{Ar}$ & $\mathrm{Kr}$ & $\mathrm{Ne}$ & $\mathrm{Rn}$ & $\mathrm{CBr_4}$ & $\mathrm{C_6H_{12}}$ & $\mathrm{H_2S}$ & $\mathrm{N_2}$ \\
\hline
$1,80$ & $1,80$ & $1,80$ & $1,82$ & $1,87$ & $1,95$ & $1,98$ & $1,99$ & $2,00$ \\
\hline
$\mathrm{C_6H_6}$ & $\mathrm{Br_2}$ & $\mathrm{CH_4}$ & $\mathrm{I_2}$ & $\mathrm{CCl_4}$ & $\mathrm{C_7H_{14}}$ & $\mathrm{H_2O}$ & $\mathrm{H_2}$ & $\mathrm{Cl_2}$ \\
\hline
$2,02$ & $2,08$ & $2,11$ & $2,14$ & $2,22$ & $2,32$ & $2,37$ & $2,39$ & $2,42$ \\
\hline
$\mathrm{CF_4}$ & $\mathrm{C_8H_{18}}$ & $\mathrm{F_2}$ & $\mathrm{C_3H_6}$ & $\mathrm{O_2}$ & $\mathrm{C_5H_10}$ & $\mathrm{O=C=Me_2}$ & $\mathrm{P_2}$ & $Pt$ \\
\hline
$2,55$ & $2,63$ & $2,70$ & $2,73$ & $2,86$ & $2,86$ & $2,86$ & $3,05$ & $3,15$ \\
\hline
$\mathrm{He}$ & $\mathrm{W}$ & $\mathrm{S_2}$ & $\mathrm{C_2H_6}$ & $\mathrm{Ag}$ & $\mathrm{Au}$ & $\mathrm{Zn}$ & $\mathrm{Mo}$ & $\mathrm{Cu}$ \\
\hline
$3,25$ & $3,25$ & $3,35$ & $3,41$ & $3,49$ & $3,61$ & $3,74$ & $3,85$ & $3,97$ \\
\hline
$\mathrm{Cd}$ & $\mathrm{Zr}$ & $\mathrm{Pb}$ & $\mathrm{Rb}$ & $\mathrm{Na}$ & $\mathrm{K}$ & $\mathrm{Cs}$ & $\mathrm{Hg}$ & $\mathrm{Li}$ \\
\hline
$4,18$ & $4,21$ & $6,61$ & $6,74$ & $6,75$ & $6,78$ & $7,78$ & $7,53$ & $8,43$ \\
\hline
\end{tabular}}

\end{minipage}
\begin{minipage}[top]{.45\textwidth}
кой температуры: $RT/\rho M = 2Z_2L\rho/(ZM)$. Отсюда следует
\begin{center}
$T{\textit{\text{\tiny \textup{кр2}}}} = \frac{\mbox{$2Z_2L$}}{\mbox{$Z_0R$}}$
\end{center}

Если разделить критическую температуру $T{\textit{\text{\tiny \textup{кр2}}}}$ на критическю температуру $T{\textit{\text{\tiny \textup{кр1}}}}$, то получится безразмерное число: $T{\textit{\text{\tiny \textup{кр2}}}}/T{\textit{\text{\tiny \textup{кр1}}}} \thickapprox i - 2$. Заметим, что при $i = 3$, т. е. для веществ с одноатомными молекулами, это отношение равно единице, значит, разные подходы дают один и тот же результат для критической температуры.
\end{minipage}
\begin{multicols}{2}
\noindent
цепочек атомов (молекул), которые могут изгибаться, и междуд цепочками имеются промежутки. Если для щелочных металлов $Z_2 \thickapprox 7$, то это означает, что значительня часть вещества в среднем по времени пребывает в составе неких трубок-цилиндров-струй с заполнением. На каждую молекулу, находящуюся на поверхности такой трубки, приходится по шесть таких же соседок, живущих на поверхности, и одна - внутри трубки. А в промежутках между трубками - пустота. Это можно представить как структуру пены с тонкими стенками и толстыми участками, на которых стенки соединяются друг с другом под неким углом в пространстве или как модель кристаллической решетки, у которой стерженьки, соединяющие узлы решетки, являются теми самыми цилиндрами-трубками-струями.

Каждая молекула, входящая в состав кондесированного вещества, при невысоких внешних давления и при температуре значительно меньше критической занимает объем, примерно $D^3$. А кинетическая энергия поступательного движения молекулы равна  
$E{\textit{\text{\scriptsize \textup{кин}}}}=3kT/2$. Таким образом, давление, связанное с тепловым движением, которое молекула оказывает на стенки своей <<ячейки/клетки>>, равно $(2/3) E{\textit{\text{\scriptsize \textup{кин}}}}/D^3=kT/D^3=RT/\rho M$. Когда эта величина сравняется с собственным давлением вещества при числе соседок $Z_2$, вещество не сможет находиться в конденсированном состоянии, т.е. это равенство дает еще один критерий нахождения критичес

\textbf{Поверхностная энергия и коэффициент поверхностного натяжения}

Молекулы кондесированного вещества, которые живут на границе раздела с паром этого же вещества, имеют меньшее количество ближайших соседок $Z_3$, чем молекулы, живущие внутри объема и имеющие число соседок $Z$. Поэтому потенциальная яма, в которой находится каждая молекула на поверхности, имеет меньшую глубину.

При невысоких температурах, когда взаимодействием молекулы на поверхности с молекулами пара можно пренебречь, у каждой молекуы на поверхности в среднем число соседок меньше на определенную долю от максимально числа соседок: $\triangle Z = Z-Z_3$. Например, если $Z = 12$ при плотной упаковке шариков, то у молекул на плоской поверхности соседо всего $Z_3 = 9$. Следовательно, потенциальная энергия в положении равновесия у таких молекул равна $-9U_0$, а избыточная энергия равна $3U_0 = ZU_0/4$. Каждая молкула на поверхности занимает площадь, которая по порядку величины равна $D^2$. Для самой плотной упаковки шариков эта площадь составляет $\sqrt 3D^2/2$. Поэтому избыточная энергия, приходящаяся на единицу площади, равна примерно
\begin{center}
$\sigma_0 = \frac{\mbox{$ZU_0$}}{\mbox{$2\sqrt3D^2$}}$.
\end{center}
Это и есть коэффициент поверхностного натяжения при невысоких температурах.

По мере роста температуры давление и плотность насыщенного пара растут, и молекулы пара создают для каждой молекулы,
% This line here is a comment. It will not be printed in the document.
\end{multicols}
\end{document}