\documentclass[14pt]{article}
\usepackage{geometry}
\usepackage[utf8]{inputenc}
\usepackage{cmap}
\usepackage[russian]{babel}
\usepackage{textcomp}
\usepackage{indentfirst}
\usepackage{amssymb}
\usepackage{graphicx}
\usepackage{lipsum}
\usepackage{setspace}
\usepackage{extsizes}
\usepackage{enumitem}
\usepackage{tabularx}
\usepackage{amsmath}
\usepackage{tocloft}
\usepackage{minted}
\usepackage[colorlinks=true,urlcolor=blue,linkcolor=]{hyperref}

\renewcommand{\cftsecleader}{\cftdotfill{\cftdotsep}}
\renewcommand{\cfttoctitlefont}{\LARGE\bfseries}
% Начало блока кода, который позволяет включить section* в оглавление
\usepackage{etoolbox}
\makeatletter
\newcommand\mysection[1]{%
	  \addcontentsline{toc}{section}{#1}%
	  \section*{#1}%
}
\makeatother

\makeatletter
\newcommand\mysubsection[1]{%
	  \addcontentsline{toc}{subsection}{#1}%
	  \subsection*{#1}%
}
\makeatother

\makeatletter
\newcommand\mysubsubsection[1]{%
	  \addcontentsline{toc}{subsubsection}{#1}%
	  \subsubsection*{#1}%
}
\makeatother
% Конец блока кода

\usepackage{titlesec}
\titleformat*{\section}{\LARGE\bfseries\centering}
\titleformat*{\subsection}{\Large\it}
\usepackage{pgfplots}

\newcolumntype{Y}{>{\centering\arraybackslash}X}

\begin{document}
\newgeometry{top=0.1in,bottom=0in,right=0.1in,left=0.1in}
\begin{spacing}{1}
\begin{center}
	\makebox[\linewidth][s]{МИНИСТЕРСТВО НАУКИ И ВЫСШЕГО ОБРАЗОВАНИЯ РОССИЙСКОЙ ФЕДЕРАЦИИ} \\
	\vspace{5mm}
	ФЕДЕРАЛЬНОЕ ГОСУДАРСТВЕННОЕ АВТОНОМНОЕ ОБРАЗОВАТЕЛЬНОЕ \\ УЧРЕЖДЕНИЕ ВЫСШЕГО ОБРАЗОВАНИЯ  \\
	\guillemotleft Национальный исследовательский университет ИТМО\guillemotright \\
	\vspace{5mm}
	ФАКУЛЬТЕТ ПРОГРАММНОЙ ИНЖЕНЕРИИ И КОМПЬЮТЕРНОЙ ТЕХНИКИ
	\vspace{60mm}
	
	{\bf \LARGE ЛАБОРАТОРНАЯ РАБОТА №2} \\
	{ \Large по дисциплине \\
	\guillemotleft Вычислительная математика\guillemotright \\
	Вариант № 13}\\
\end{center}
\vspace{50mm}

\begin{flushright}
	{\it \textbf{Выполнил работу:}}\\
	Студент группы P3218 \\
	Рамеев Тимур Ильгизович \\
	{\it \textbf{Преподаватель:}}\\
	Бострикова Дарья \\
	Константиновна \\
\end{flushright}
\vspace{18mm}
\end{spacing}
\begin{center}
    Санкт-Петербург  2024
\end{center}

\newgeometry{a4paper, top=10mm, bottom=20mm, left=20mm, right=10mm}

\newpage
\begin{center}
	\tableofcontents 
\end{center}
\setcounter{page}{1}

\newpage
\mysection{Цель}
	Изучить численные методы решения нелинейных уравнений и их систем, найти корни заданного нелинейного уравнения/системы нелинейных уравнений, выполнить программную реализацию методов.
\mysection{Задание}
	\mysubsection{Для вычислительной реализации задачи}
	\underline{\bf 1 Часть. Решение нелинейного уравнения}
	\begin{enumerate}[itemsep=-5pt]
		\item Отделить корни заданного нелинейного уравнения графически (вид уравнения представлен в табл. 6)
		\item Определить интервалы изоляции корней.
		\item Уточнить корни нелинейного уравнения (см. табл. 6) с точностью $\epsilon =10^{-2}$
		\item Используемые методы для уточнения каждого из 3-х корней многочлена представлены в таблице 7.
		\item Вычисления оформить в виде таблиц (1-5), в зависимости от заданного метода. Для всех значений в таблице удержать 3 знака после запятой.
		\begin{enumerate}[label=\theenumi.\arabic*, itemsep=0pt]
			\item Для метода половинного деления заполнить таблицу 1.
			\item Для метода хорд заполнить таблицу 2.
			\item Для метода Ньютона заполнить таблицу 3.
			\item Для метода секущих заполнить таблицу 4.
			\item Для метода простой итерации заполнить таблицу 5. Проверить условие сходимости метода на выбранном интервале.
		\end{enumerate}
		\item Заполненные таблицы отобразить в отчете. 
	\end{enumerate}

	\underline{\bf 2 Часть. Решение системы нелинейных уравнений}
	\begin{enumerate}[itemsep=-5pt]
		\item Отделить корни заданной системы нелинейных уравнений графически (вид системы представлен в табл. 8).
		\item Используя указанный метод, решить систему нелинейных уравнений с точностью до 0,01.
		\item Для метода простой итерации проверить условие сходимости метода.
		\item Подробные вычисления привести в отчете.
	\end{enumerate}

\newpage
	\mysubsection{Для программной реализации задачи}
	\underline{\bf 1 Часть. Решение нелинейного уравнения}
	\begin{enumerate}[itemsep=-5pt]
		\item Все численные методы (см. табл. 9) должны быть реализованы в виде отдельных подпрограмм/методов/классов.
		\item Пользователь выбирает уравнение, корень/корни которого требуется вычислить (3-5 функций, в том числе и трансцендентные), из тех, которые предлагает программа.
		\item Предусмотреть ввод исходных данных (границы интервала/начальное приближение к корню и погрешность вычисления) из файла или с клавиатуры по выбору конечного пользователя.
		\item Выполнить верификацию исходных данных. Необходимо анализировать наличие корня на введенном интервале. Если на интервале несколько корней или они отсутствуют – выдавать соответствующее сообщение. Программа должна реагировать на некорректные введенные данные.
		\item Для методов, требующих начальное приближение к корню (методы Ньютона, секущих, хорд с фиксированным концом, простой итерации), выбор начального приближения (а или b) вычислять в программе.
		\item Для метода простой итерации проверять достаточное условие сходимости метода на введенном интервале.
		\item Предусмотреть вывод результатов (найденный корень уравнения, значение функции в корне, число итераций) в файл или на экран по выбору конечного пользователя.
		\item Организовать вывод графика функции, график должен полностью отображать весь исследуемый интервал (с запасом).
	\end{enumerate}

	\underline{\bf 2 Часть. Решение системы нелинейных уравнений}
	\begin{enumerate}[itemsep=-5pt]
		\item Пользователь выбирает предлагаемые программой системы двух нелинейных уравнений (2-3 системы).
		\item Организовать вывод графика функций.
		\item Начальные приближения ввести с клавиатуры.
		\item Для метода простой итерации проверить достаточное условие сходимости.
		\item Организовать вывод вектора неизвестных: $x_1, x_2$.
		\item Организовать вывод количества итераций, за которое было найдено решение.
		\item Организовать вывод вектора погрешностей: $|x_i^{(k)} - x_i^{(k - 1)}|$
		\item Проверить правильность решения системы нелинейных уравнений.
	\end{enumerate}

\newpage
\mysection{Рабочие формулы используемых методов}
	\mysubsection{Метод половинного деления}
	$x_i = \frac{a_i + b_i}{2}$
	\mysubsection{Метод хорд}
	$x_i = \frac{a_if(b_i) + b_i f(a_i)}{f(b_i) - f(a_i)}$
	\mysubsection{Метод Ньютона}
	$x_i = x_{i-1}-\frac{f(x_{i-1})}{f'(x_{i-1})}$
	\mysubsection{Метод секущих}
	$x_i = x_{i}-\frac{x_i - x_{i - 1}}{f(x_i) - f(x_{i-1})}f(x_i)$
	\mysubsection{Метод простой итерации}
	$x_i = \phi(x_{i-1})$

\newpage
\mysection{Решение нелинейного уравнения}
	\mysubsection{График функции}
	\begin{tikzpicture}
	    \begin{axis}[
	        xlabel={$x$},
	        ylabel={$y$},
	        xmin=-10, xmax=10,
	        ymin=-10, ymax=100,
	        grid=both,
		legend style={at={(0.5,1.05)},anchor=south}, % Помещаем легенду над графиком
	        axis lines=middle,
	        width=\textwidth,
	        height=\textwidth,
	    ]
	    \addplot[domain=-10:10, samples=100, smooth, thick, red, solid] {x^3+4.81*x^2-17.37*x+5.38};
	    \addplot[black, only marks, mark=*] coordinates {
       		 (-7.293,0)
 		(0.345,0) % точки пересечения с осью x
 		(2.138,0)
   	   };
	    \legend{$f(x) = x^3+4.81x^2-17.37x+5.38$}	
	    \node[above] at (axis cs: -8, 0) {$a_1$};
	    \node[above] at (axis cs: -6,0) {$b_1$};
	    \node[above] at (axis cs: 0,0) {$a_2$};
	    \node[above] at (axis cs: 2,0) {$b_2   a_3$};
	    \node[above] at (axis cs: 4,0) {$b_3$};
	    \end{axis}
	\end{tikzpicture}

	\newpage
	\mysubsection{Определение промежутков изоляции корней}
		Определим интервалы изоляции корней по графику "на глаз" 
		\begin{enumerate}
			\item{Крайний левый корень - $[-8 : -6]$}
			\item{Центральный корень - $[0 : 2]$}
			\item{Крайний правый корень - $[2 : 4]$}
		\end{enumerate}
		\mysubsection{Метод хорд}
		\begin{table}[htbp]
		    \centering
		    \begin{tabularx}{\textwidth}{|c|Y|Y|Y|Y|Y|Y|Y|}
		        \hline
		        № Шага & $a$ & $b$ & $x$ & $f(a)$ & $f(b)$ & $f(x)$ & $|x_{k+1} - x_{k}|$\\
		        \hline
			1 & -8,000 & -6,000 & -7,055 & -59,820 & 66,760 & 16,185 &  - \\
		        \hline
			2 & -8,000 & -7,055 & -7,256 & -59,820 & 16,185 & 2,636 &  0,201 \\
		        \hline
			3 & -8,00 & -7,256 & -7,287 & -59,820 & 2,636  & 0,426 & 0,031 \\
		        \hline
			4 & -8,00 & -7,287 & -7,292 & -59,820 &  0,426 & 0.066 & 0,005 \\
		        \hline
		    \end{tabularx}
		    \caption{Для крайне левого корня}
		    \label{tab:2}
		\end{table}
	\mysubsection{Метод Ньютона}
		Найдем производную исходной функции:
		\begin{center}
			$f'(x) = 3x^2 + 9,62x - 17,37$
		\end{center}
		\begin{table}[htbp]
		    \centering
		    \begin{tabularx}{\textwidth}{|Y|Y|Y|Y|Y|Y|}
		        \hline
		        № Итерации & $x_{k}$& $f(x_{k})$ & $f'(x_{k})$ & $x_{k+1}$ & $|x_{k+1} - x_{k}|$\\
		        \hline
			1 & 1,000 & -6,180 & -4,750 & -0,301 & 1,301\\
		        \hline
			2 & -0,301 & 11,017 & -19,994 & 0,250 & 0,551\\
		        \hline
			3 & 0,250 & 1,354 & -14,778 & 0,342 & 0,092 \\
		        \hline
			4 & 0,342 & 0,042 &-13,729 & 0,345 & 0,003 \\
			\hline
		    \end{tabularx}
		    \caption{Для центрального корня}
		    \label{tab:3}
		\end{table}
	\newpage
	\mysubsection{Метод простой итерации}
		\underline{\bf Проверка сходимости} \\
		Преобразуем уравнение к виду $x = \phi(x)$ \\
		$\phi(x) = \frac{x^3+4,81x^2+5,38}{17,37}$ \\
		$\phi'(x) = \frac{3x^2+9,62x}{17,37}$ \\
		$|\phi'(2)| \approx 1,9 > 1$ - условие сходимости не выполняется \\
		Попробуем выразить $x$ другим способом \\
		$\phi(x) = \sqrt{\frac{17,37x - x^3 - 5,38}{4,81}}$	\\
		$|\phi'(2)| \approx 2,1 > 1$ - условие сходимости не выполняется \\
		Попробуем выразить $x$ другим способом \\
		$\phi(x) = \sqrt[3]{17,37x - 4,81x^2 - 5,38}$	\\
		$|\phi'(2)| \approx 2,2 > 1$ - условие сходимости не выполняется \\
		Выразить $x$ так, чтобы коэффициент сжатия был меньше единицы не получается, поэтому применим прием введения параметра $\lambda$ \\
		\begin{center}
			$f'(x) = 3x^2 + 9,62x - 17,37$ \\
			$f'(2) =13,87 \hspace{2cm} f'(4) = 69,11 \hspace{2cm} \lambda = - \frac{1}{\max_{[a, b]}|f'(x)|}  = - \frac{1}{69,11}$ \\
			$ \phi(x) = x + \lambda f(x) =x - \frac{1}{69,11}(x^3+4.81x^2-17.37x+5.38)$ \\
			$ \phi(x) = -0,014x^3 - 0,070x^2 + 1,251x - 0,078 \hspace{2cm} x_0 = 3$
		\end{center}

		\underline{\bf Решение}
		\begin{table}[htbp]
		    \centering
		    \begin{tabularx}{\textwidth}{|Y|Y|Y|Y|Y|}
		        \hline
		        № Итерации & $x_{k}$ & $x_{k+1}$ & $f(x_{k+1})$ & $|x_{k+1} - x_{k}|$\\
		        \hline
			1 & 3 & 2,667 & 12,237 & 0,333 \\
		        \hline
			2 &2,667 & 2,495 & 7,516 & 0,172  \\
		        \hline
			3 & 2,495 & 2,390 & 4,993 &0,105 \\
		        \hline
			4 & 2,390 & 2,320 & 3,458 &0,070 \\
		        \hline
			5 & 2,320 & 2,273 & 2,493 &0,053 \\
		        \hline
			6 & 2,273 & 2,239 & 1,826 &0,034 \\
		        \hline
			7 & 2,239 & 2,215 & 1,372 &0,024 \\
		        \hline
			8 & 2,215 & 2,197 & 1,040 &0,018 \\
		        \hline
			9 & 2,197 & 2,184 & 0,804 &0,013 \\
		        \hline
			10 & 2,184 & 2,174 & 0,626 &0,010 \\
		        \hline
		    \end{tabularx}
		    \caption{Для крайне правого корня}
		    \label{tab:1}
		\end{table}

\newpage
\mysection{Решение системы нелинейных уравнений}
	\mysubsection{График функций}
		\begin{tikzpicture}
		\begin{axis}[
			axis lines=middle,
			xlabel={$x$},
			ylabel={$y$},
			xmin=-0.4, xmax=1.8,
			ymin=-1.4, ymax=1.4,
			xtick={-0.785398, 0, 0.785398, 1.5708, 2.35619},
			xticklabels={$-\frac{\pi}{4}$, $0$, $\frac{\pi}{4}$, $\frac{\pi}{2}$, $\frac{3\pi}{4}$},
			ytick={-1.5708, -0.785398, 0, 0.785398, 1.5708},
			yticklabels={$-\frac{\pi}{2}$, $-\frac{\pi}{4}$, $0$, $\frac{\pi}{4}$, $\frac{\pi}{2}$},
			grid=both,
			axis lines=middle,
			width=\textwidth,
			height=\textwidth,
			legend style={at={(0.5,1.05)},anchor=south}, % Помещаем легенду над графиком
		    ]
		    	\addplot[blue, thick, domain=0.5:1.5, samples=100] {rad(asin(2 - 2*x))};
			\addlegendentry{$\sin(y) + 2x = 2$}
		    	\addplot[red, thick, domain=-1:2, samples=100] {0.7 - cos(deg(x - 1))};
			\addlegendentry{$y + \cos(x - 1) = 0,7$}
			\addplot[black, only marks, mark=*] coordinates {
	       		 (1.143,-0.29)
   	  		 };
		\end{axis}
		\end{tikzpicture}
	\mysubsection{Определение областей изоляции корней}
		По графику видно, что искомый корень находится в области:
		\begin{equation*}
			\begin{cases}
				0 < x < \frac{\pi}{2} \\
				- \frac{\pi}{4} < y < 0 

			\end{cases}
		\end{equation*}
	\mysubsection{Метод простой итерации}
		\underline{\bf Проверка сходимости} \\
		\begin{equation*}
			\begin{cases}
				x = 1 - \frac{\sin(y)}{2} \\
				y = 0,7 - \cos(x - 1)
			\end{cases}
		\end{equation*}
		\begin{center}
			$\frac{\partial \phi_1}{\partial x} = 0$ \hspace{3cm}
			$\frac{\partial \phi_1}{\partial y} =- \frac{\cos(y)}{2}$ \\
			$\frac{\partial \phi_2}{\partial x} = \sin(x - 1)$ \hspace{3cm}
			$\frac{\partial \phi_2}{\partial y} = 0$ \\
		\end{center}
		$|\frac{\partial \phi_1}{\partial x}| + |\frac{\partial \phi_1}{\partial y}| < 1$ \\
		$|\frac{\partial \phi_2}{\partial x}| + |\frac{\partial \phi_2}{\partial y}| < 1$ - следовательно процесс сходящийся

		\vspace{5mm}
		
		\underline{\bf Решение} \\
		Выберем начальное приближение  $x^{(0)} = \frac{\pi}{2} = 1,571; y^{(0)} = - \frac{\pi}{4} = -0,785$ 
		\begin{table}[htbp]
		    \centering
		    \begin{tabularx}{\textwidth}{|Y|Y|Y|Y|Y|Y|}
		        \hline
		        № Шага & $x^{(i)}$& $y^{(i)}$ & $|x^{(i)} - x^{(i-1)}|$ & $|y^{(i)} - y^{(i-1)}|$\\
		        \hline
			1 & 1,354 & -0,141 &  0,217 & 0,644 \\
		        \hline
			2 & 1,070 & -0,238 &  0,284 & 0,097 \\ 
		        \hline
			3 & 1,118 & -0,298 & 0,048 & 0,060 \\
		        \hline
			4 & 1,147 & -0,293 & 0,029 & 0,005 \\
		        \hline
			5 & 1,144 & -0,289 & 0,003 &  0,004 \\
		        \hline
		    \end{tabularx}
		    \caption{Система нелинейных уравнений}
		    \label{tab:4}
		\end{table}
		\vspace{5mm}

\newpage
\mysection{Программная реализация задачи}
	\mysubsection{Ссылка на исходный код к лабораторной работе}
	\href{https://github.com/Stt1xX/Studies/tree/main/Computational%20mathematics/Lab2}{Git Hub}
	\mysubsection{Основные методы программы}
\underline{\bf Метод половинного деления}
\begin{minted}[mathescape, linenos]{python}
def half_division(left, right, equation, accuracy):
    counter = 0
    search_value = define_method(equation)
    while right - left > accuracy:
        x = (right + left) / 2
        if search_value(x) * search_value(left) > 0:
            left = x
        else:
            right = x
        counter += 1
    return {"status" : 1, "root" : x, "value" : search_value(x),
 		"number_of_iterations" : counter}
\end{minted}
\newpage
\underline{\bf Метод секущих}
\begin{minted}[mathescape, linenos]{python}
def secant(left, right, approach, equation, accuracy):
    counter = 0
    search_value = define_method(equation)
    search_value_second_derivative = 
define_method_for_second_derivative(equation)
    preprevious = None
    previous = None
    if search_value(left) * search_value_second_derivative(left) > 0:
        preprevious = left
    else:
        preprevious = right

    if approach != "" and abs(preprevious - approach) > accuracy:
        previous = approach
    elif preprevious == left:
        previous = preprevious + 0.2
    else:
        previous = preprevious - 0.2
            
    while abs(preprevious - previous) > accuracy:
        x = previous - (previous - preprevious) * search_value(previous) 
/ (search_value(previous) - search_value(preprevious))
        counter += 1
        preprevious = previous
        previous = x

    return {"status" : 1, "root" : previous, "value" :
 search_value(preprevious), "number_of_iterations" : counter}
\end{minted}
\newpage
\underline{\bf Метод простых итераций}
\begin{minted}[mathescape, linenos]{python}
def simple_iterations_equation(left, right, approach, equation, accuracy):
    search_value = define_method(equation)
    search_first_derivative = define_method_for_first_derivative(equation)
    counter = 1
    x = None
    previous = None
    if approach != "":
        previous = approach
        if search_first_derivative(previous) > 0:
            gamma = -1 / search_max_derivative(left, right, equation)
.get('max_value')
        else:
            gamma = 1 / search_max_derivative(left, right, equation)
.get('max_value')
    else:
        if abs(search_first_derivative(left)) >=
 abs(search_first_derivative(right)):
            previous = left
        else:
            previous = right
        if search_first_derivative(previous) > 0:
            gamma = -1 / search_max_derivative(left, right, equation)
.get('max_value')
        else:
            gamma = 1 / search_max_derivative(left, right, equation)
.get('max_value')
    x = gamma * search_value(previous) + previous
    if abs(gamma * search_first_derivative(left) + 1) > 1
 or  abs(gamma * search_first_derivative(right) + 1) > 1:
        return {"status" : 0, "error" : "К сожалению этот метод не сходится, 
попробуйте уменьшить промежуток поиска корня"}
    while abs(x - previous) > accuracy:
        previous = x
        x = gamma * search_value(previous) + previous
        counter += 1
    return {"status" : 1, "root" : x, "value" : search_value(x), 
"number_of_iterations" : counter}
\end{minted}
\mysection{Вывод}
В ходе выполнения лабораторной работы было изучено несколько итерационных методов решения нелинейных уравнений, а также два метода для решения системы нелинейных уравнений. 
Была реализована программа на языках программирования Python и Java Script. Были
проанализированы условия возможности применения тех или иных численных
методов, их достоинства и недостатки. 

Анализ методов

Метод Половинного деления - прост в реализации, однако имеет линейную сходимость.

Метод Хорд - прост в реализации. Порядок сходимости метода хорд выше, чем у
метода половинного деления.

Метод Ньютона - использует касательную для нахождения корня. Скорость
сходимости: быстрая, квадратичная.

Метод Половинного деления - скорость сходимости зависит от выбора функции $ \phi(x)$ и начального
приближения.

Метод Секущих - имеет меньший объем вичислений по сравнению с методом Ньютона, т. к. не нужно вычислять производную, однако порядок сходимости ниже, чем у метода касательных и равен золотому сечению.
\end{document}
