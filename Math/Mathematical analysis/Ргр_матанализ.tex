\documentclass[12pt, a4paper]{article}

\usepackage[utf8]{inputenc}
\usepackage[english, russian]{babel}
\usepackage{fancyhdr}
\usepackage{amsmath}
\usepackage{amsthm}
\usepackage{float}
\usepackage{graphicx}
\usepackage{pgfplots}
\usepackage{float}
\usepackage{xcolor}
\pgfplotsset{width=\textwidth, compat=1.13}

\usepgfplotslibrary{external}
\usepgfplotslibrary{fillbetween}



\graphicspath{{./}}
\newcommand{\Mod}[1]{\ \mathrm{mod}\ #1}

\usepackage[a4paper, margin=1.5cm]{geometry}

\usepackage{titlesec}
\titlelabel{\thetitle.\quad}

\pagestyle{plain}

\fancypagestyle{firstpage}{%
  \chead{
  МИНИСТЕРСТВО НАУКИ И ВЫСШЕГО ОБРАЗОВАНИЯ РОССИЙСКОЙ ФЕДЕРАЦИИ 
ФЕДЕРАЛЬНОЕ ГОСУДАРСТВЕННОЕ АВТОНОМНОЕ  
ОБРАЗОВАТЕЛЬНОЕ УЧРЕЖДЕНИЕ ВЫСШЕГО ОБРАЗОВАНИЯ\bigskip

«Национальный исследовательский университет ИТМО»\bigskip

ФАКУЛЬТЕТ ПРОГРАММНОЙ ИНЖЕНЕРИИ И КОМПЬЮТЕРНОЙ ТЕХНИКИ 
}
\fancyfoot[CO]{Санкт-Петербург, 2023}%
}



\definecolor{aqua}{HTML}{003844}
\definecolor{peri}{HTML}{5EB1BF}
\definecolor{royal_blue}{HTML}{0A2463}
\definecolor{periwinkle}{HTML}{D8DCFF}
\definecolor{cerulean}{HTML}{247BA0}
\definecolor{bloodred}{HTML}{690500}
\definecolor{imperial_red}{HTML}{FB3640}
\definecolor{purple}{HTML}{511730}
\definecolor{tangerine}{HTML}{FFA781}

\newtheorem*{task}{Условие}
\newtheorem*{finish}{Заключение}

\tikzexternalize
\begin{document}
\newgeometry{top=1.6cm,bottom=1.6cm, left = 1.2cm, right = 1.2cm}

\topskip0pt
\vspace*{0.25\textheight}
\begin{center}
\textbf{\LARGE РАСЧЁТНО-ГРАФИЧЕСКАЯ РАБОТА}

\LARGE по теме

\LARGE <<Интеграл функции одной переменной>>

\LARGE по дисциплине

\LARGE <<Математика>>\bigskip

\LARGE Вариант № 8
\end{center}
\vspace*{5cm}
\begin{flushright}
\begin{minipage}{.33\linewidth}
\textit{\textbf{Выполнили:}}\\
\begin{tabular}{l l @{\hspace{8pt}-\hspace{8pt}} l l}
Хороших & Дмитрий & P3117 & 1.5\\
Рамеев & Тимур & P3118 & 1.5
\end{tabular}
\textit{\textbf{Преподаватель:}}\\
Правдин Константин
\end{minipage}
\end{flushright}


\thispagestyle{firstpage}
\newpage
\tableofcontents

\restoregeometry
\section{1-е Задание}
\begin{task}
Исследуйте интегральную сумму функции $\frac{1}{\sqrt{1-x^2}}$ на отрезке $\left[-\frac{\sqrt{3}}{2};\frac{\sqrt{3}}{2}\right]$.
\end{task}

\subsection{Интегральная сумма}
Начнём исследование интегральной суммы функции со сравнения площади криволинейной трапеции, ограниченной графиком функции $\frac{1}{\sqrt{1-x^2}}$ на отрезке $\left[-\frac{\sqrt{3}}{2};\frac{\sqrt{3}}{2}\right]$, с площадью ступенчатой фигуры (изображающей интегральную сумму на плоскости). На рисунках \ref{gr:mid5} - \ref{gr:mid11} светлым цветом выделена криволинейная трапеция,а тёмным - ступени фигуры.

Рассмотрим рисунок \ref{gr:mid5}, на котором ступенчатая фигура построена на 5-и элементарных отрезках, а $c_i$, определяющее высоту ступени по значению $f(c_i)$, установлено в середину каждого элементарного отрезка. Можно заметить, что площадь ступенчатой фигуры слабо совпадает с площадью трапеции - некоторые участки трапеции не покрыты фигурой, а некоторые участки фигуры выходят за пределы трапеции. Очевидно, что такое приблежение не очень точно.


\newcommand{\stair}[2]{
%x2
\addplot [
        thick,
        color=\scolor,
        fill=\scolor, 
        fill opacity=0.5,
        domain =\st + #1*\sz:\st + (#1+1)*\sz,
        name path = p#1
    ]{#2};
\node[circle, \scolor, fill, inner sep=1pt, thick] at (\st + #1*\sz, 0){};
\node[circle, \scolor, fill, inner sep=1pt, thick] at (\st + (#1*\sz + 1*\sz, 0){};
\draw [dotted, \scolor,  thick] (\st + #1*\sz + \sz*\npos, 0) node[rectangle, fill, inner sep=1.5pt]{} -- (\st + #1*\sz + \sz*\npos, #2) node[rectangle, fill, inner sep=1.5pt]{};

\path[name path = axis#1] (\st + #1*\sz , 0) -- (\st + #1*\sz + 1*\sz , 0);
\addplot [
        thick,
        color=\scolor,
        fill=\scolor, 
        fill opacity=0.2
    ]
    fill between[
        of=p#1 and axis#1,
        soft clip={domain=\st+#1*\sz:\st + #1*\sz+1*\sz},
    ];
}

\def\pcolor{peri}
\def\scolor{royal_blue}
%5 splits - mid
\begin{figure}[H]
\begin{tikzpicture}
\begin{axis}[
	axis lines = left,
	xlabel = \(x\),
	ylabel = {\(f(x)\)},
	ymin=-1,
	xmin=-1.8,
	xmax=1.8,
	grid=both,
    grid style={line width=.1pt, draw=gray!10},
    major grid style={line width=.2pt,draw=gray!50},
    minor tick num=5,
	axis x line = bottom,
	axis x line shift = -1,
	ymax=5,
	xtick distance={0.5},
	extra x ticks={ -0.87, 0.87},
	extra x tick labels={$-\frac{\sqrt{3}}{2}$, $\frac{\sqrt{3}}{2}$},
	extra x tick style ={
	grid = none},
	legend style = {row sep =0.5cm}
	]

\addplot[
	ultra thick,
    domain=-1:1,
    samples=100,
    color=\pcolor,
    restrict y to domain=-20:20,
    name path = f,
]
{1/(1-x^2)^0.5};
\draw [dashed, \pcolor, ultra thick] (-0.87,0) -- (-0.87,2);
\draw [dashed, \pcolor, ultra thick] (0.87,0) -- (0.87,2);
\path[name path = axis] (-0.87,0) -- (0.87,0);

\addplot [
        thick,
        color=\pcolor,
        fill=\pcolor, 
        fill opacity=0.2
    ]
    fill between[
        of=f and axis,
        soft clip={domain=-0.87:0.87},
    ];

\def\sz{0.348}
\def\st{-0.87}
\def\npos{0.5}
\stair{0}{1.393}
\stair{1}{1.067}
\stair{2}{1.0}
\stair{3}{1.067}
\stair{4}{1.393}
\legend{$\frac{1}{\sqrt{1-x^2}}$,,,$\overline{S}_i$}
\end{axis}
\end{tikzpicture}

\caption{$f(x)$ со ступенчатой фигурой: $n = 5$, $c_i$ - середина отрезка}
\label{gr:mid5}
\end{figure}
%5 splits-left
\begin{figure}
\begin{minipage}{.5\textwidth}
\begin{tikzpicture}
\begin{axis}[
	axis lines = left,
	xlabel = \(x\),
	ylabel = {\(f(x)\)},
	ymin=-1,
	xmin=-1.8,
	xmax=1.8,
	grid=both,
    grid style={line width=.1pt, draw=gray!10},
    major grid style={line width=.2pt,draw=gray!50},
    minor tick num=5,
	axis x line = bottom,
	axis x line shift = -1,
	ymax=5,
	xtick distance={0.5},
	extra x ticks={ -0.87, 0.87},
	extra x tick labels={$-\frac{\sqrt{3}}{2}$, $\frac{\sqrt{3}}{2}$},
	extra x tick style ={
	grid = none},
	legend style = {row sep =0.5cm}
	]

\addplot[
	ultra thick,
    domain=-1:1,
    samples=100,
    color=\pcolor,
    restrict y to domain=-20:20,
    name path = f,
]
{1/(1-x^2)^0.5};
\draw [dashed, \pcolor, ultra thick] (-0.87,0) -- (-0.87,2);
\draw [dashed, \pcolor, ultra thick] (0.87,0) -- (0.87,2);
\path[name path = axis] (-0.87,0) -- (0.87,0);

\addplot [
        thick,
        color=\pcolor,
        fill=\pcolor, 
        fill opacity=0.2
    ]
    fill between[
        of=f and axis,
        soft clip={domain=-0.87:0.87},
    ];
\def\sz{0.348}
\def\st{-0.87}
\def\npos{0.0}
\stair{0}{2.028}
\stair{1}{1.172}
\stair{2}{1.015}
\stair{3}{1.015}
\stair{4}{1.172}
\end{axis}
\end{tikzpicture}
\caption{$f(x)$ со ступенчатой фигурой:\\ $n = 5$, $c_i$ - крайнее левое положение}
\label{gr:left5}
\end{minipage}%
%
%5 splits - right
\begin{minipage}{.5\textwidth}
\begin{tikzpicture}
\begin{axis}[
	axis lines = left,
	xlabel = \(x\),
	ylabel = {\(f(x)\)},
	ymin=-1,
	xmin=-1.8,
	xmax=1.8,
	grid=both,
    grid style={line width=.1pt, draw=gray!10},
    major grid style={line width=.2pt,draw=gray!50},
    minor tick num=5,
	axis x line = bottom,
	axis x line shift = -1,
	ymax=5,
	xtick distance={0.5},
	extra x ticks={ -0.87, 0.87},
	extra x tick labels={$-\frac{\sqrt{3}}{2}$, $\frac{\sqrt{3}}{2}$},
	extra x tick style ={
	grid = none},
	legend style = {row sep =0.5cm}
	]

\addplot[
	ultra thick,
    domain=-1:1,
    samples=100,
    color=\pcolor,
    restrict y to domain=-20:20,
    name path = f,
]
{1/(1-x^2)^0.5};
\draw [dashed, \pcolor, ultra thick] (-0.87,0) -- (-0.87,2);
\draw [dashed, \pcolor, ultra thick] (0.87,0) -- (0.87,2);
\path[name path = axis] (-0.87,0) -- (0.87,0);

\addplot [
        thick,
        color=\pcolor,
        fill=\pcolor, 
        fill opacity=0.2
    ]
    fill between[
        of=f and axis,
        soft clip={domain=-0.87:0.87},
    ];
\def\sz{0.348}
\def\st{-0.87}
\def\npos{1.0}
\stair{0}{1.172}
\stair{1}{1.015}
\stair{2}{1.015}
\stair{3}{1.172}
\stair{4}{2.028}
\end{axis}
\end{tikzpicture}
\caption{$f(x)$ со ступенчатой фигурой:\\ $n = 5$, $c_i$ - крайнее правое положение}
\label{gr:right5}
\end{minipage}
\end{figure}

Изменим положение точки $c_i$ и посмотрим, как при этом изменяется фигура.

На рисунке \ref{gr:left5} ступенчатая фигура построена на 5-и элементарных отрезках, а точка $c_i$ установлена в крайнее \textbf{левое} положение каждого из отрезков. Легко видеть, что на участке 
$\left[-\frac{\sqrt{3}}{2};0\right]$, на котором $f(x)$ убывает, ступенчатая фигура \textbf{переоценивает} криволинейную трапецию и выходит за её границу, а на участке $\left[0;\frac{\sqrt{3}}{2}\right]$ наоборот - \textbf{недооценивает}.

Для ступенчатой фигуры с 5-ю элементарными отрезками и точкой $c_i$, установленной в крайнее \textbf{правое} положение каждого из отрезков, возникает противоположенная ситуация: на участке 
$\left[-\frac{\sqrt{3}}{2};0\right]$ ступенчатая фигура \textbf{недооценивает} криволинейную трапецию, а на участке $\left[0;\frac{\sqrt{3}}{2}\right]$ - \textbf{переоценивает}.

В сравнении со ступенчатой фигурой, изображённой на рисунке \ref{gr:mid5}, фигуры на рисунках \ref{gr:left5} и \ref{gr:right5} оценивает криволинейную трапеции менее точно. Таким образом, можно заключить, что оптимальное положение $c_i$ для оценки трапеции при малом числе ступеней - середина элементарного отрезка.

%11 splits

\begin{figure}[H]
\begin{tikzpicture}
\begin{axis}[
	axis lines = left,
	xlabel = \(x\),
	ylabel = {\(f(x)\)},
	ymin=-1,
	xmin=-1.8,
	xmax=1.8,
	grid=both,
    grid style={line width=.1pt, draw=gray!10},
    major grid style={line width=.2pt,draw=gray!50},
    minor tick num=5,
	axis x line = bottom,
	axis x line shift = -1,
	ymax=5,
	xtick distance={0.5},
	extra x ticks={ -0.87, 0.87},
	extra x tick labels={$-\frac{\sqrt{3}}{2}$, $\frac{\sqrt{3}}{2}$},
	extra x tick style ={
	grid = none},
	legend style = {row sep =0.5cm}
	]

\addplot[
	ultra thick,
    domain=-1:1,
    samples=100,
    color=\pcolor,
    restrict y to domain=-20:20,
    name path = f,
]
{1/(1-x^2)^0.5};
\draw [dashed, \pcolor, ultra thick] (-0.87,0) -- (-0.87,2);
\draw [dashed, \pcolor, ultra thick] (0.87,0) -- (0.87,2);
\path[name path = axis] (-0.87,0) -- (0.87,0);

\addplot [
        thick,
        color=\pcolor,
        fill=\pcolor, 
        fill opacity=0.2
    ]
    fill between[
        of=f and axis,
        soft clip={domain=-0.87:0.87},
    ];
\def\sz{0.158}
\def\st{-0.87}
\def\npos{0.5}
\stair{0}{1.634}
\stair{1}{1.291}
\stair{2}{1.136}
\stair{3}{1.054}
\stair{4}{1.013}
\stair{5}{1.0}
\stair{6}{1.013}
\stair{7}{1.054}
\stair{8}{1.136}
\stair{9}{1.291}
\stair{10}{1.634}
\legend{$\frac{1}{\sqrt{1-x^2}}$,,,$\overline{S}_i$}
\end{axis}
\end{tikzpicture}
\caption{$f(x)$ со ступенчатой фигурой:\\ $n = 11$, $c_i$ - середина отрезка}
\label{gr:mid11}
\end{figure}

Далее рассмотрим, как меняется ступенчатая фигура от изменения числа ступеней. На рисунке \ref{gr:mid11} изображена ступенчатая фигура на 11-и элементарных отрезках с точкой $c_i$, установленной в положение середины элементарного отрезка. Заметим, что в сравнении с фигурой на рисунке \ref{gr:mid5} данная фигура заметно лучше оценивает криволинейную трапецию под функцией.

При рассмотрении фигуры с числом ступеней - 7 \textit{(больше чем число ступеней фигуры с рисунка \ref{gr:mid5}, но меньше чем у фигуры с рисунка \ref{gr:mid11})} удостоверяемся, что она оценивает трапецию лучше, чем фигура на рисунке \ref{gr:mid5}, имеющая 5 ступеней, но хуже, чем фигура на рисунке \ref{gr:mid11}, имеющая 11 ступеней. В связи с этим, можно сделать очевидный вывод, что, \textbf{чем больше ступеней имеет ступенчатая фигура, тем более точно она оценивает криволинейную трапецию}.

Заметим также, что положение точки $c_i$ влияет на фигуру с 7-ю и 11-ю ступенями так же, как и на фигуры с 5-ю ступенями.

\begin{finish}
На точность оценивания криволинейной трапеции под графиком функции ступенчатой фигурой влияет:
\begin{enumerate}
\item \textbf{Положение точки $c_i$}, определяющей высоту ступени.
\item \textbf{Число ступеней}. При этом фигуры с большим числом ступеней более точно оценивают трапецию.
\end{enumerate}
\end{finish}
\subsection{Последовательность интегральных сумм}
Построим интегральную суммы функции $f(x) = \frac{1}{\sqrt{1-x^2}}$ на отрезке $\left[-\frac{\sqrt{3}}{2};\frac{\sqrt{3}}{2}\right]$:
\begin{enumerate}
\item Разобьём криволинейную трапецию на n слоёв с равными элементарными отрезками длиной:
\begin{equation*}
\Delta x_i = \Delta x = \frac{\frac{\sqrt{3}}{2} - \left(-\frac{\sqrt{3}}{2} \right)}{n} = \frac{\sqrt{3}}{n}
\end{equation*}
Условимся, что $i$-й элементарный отрезок $ = \left[x_{i-1};x_i\right] \forall i\in 1\ldots n$

При этом концы отрезков равны:
\begin{align*}
   x_{0} &= -\frac{\sqrt{3}}{2} \\
   x_{1} &= x_0 + \Delta x_i = -\frac{\sqrt{3}}{2} + \frac{\sqrt{3}}{n} \\
   x_{2} &= x_0 + 2 * \Delta x_i = -\frac{\sqrt{3}}{2} + 2 * \frac{\sqrt{3}}{n} \\
   &\vdots\\
   x_{n-1} &= x_0 + (n-1) * \Delta x_i = -\frac{\sqrt{3}}{2} + (n-1) * \frac{\sqrt{3}}{n} \\
   x_{n} &= \frac{\sqrt{3}}{2}
\end{align*}


Также введём величину $\lambda = \max_{\substack{i = 1\ldots n}} \Delta x_i$ - \textit{мелкость разбиения}
\item Приблизим площадь участка трапеции к площади прямоугольного слоя:
\begin{equation*}
\Delta S_i \approx\overline{\Delta S_i} = \Delta x_i * f(c_i)
\end{equation*}
где $\overline{\Delta S_i}$ - площадь $i$-го слоя, $\Delta S_i$ - площадь $f(x)$ на $i$-м элементарном отрезке,   \\$c_i\in \left[x_{i-1};x_i\right]$ - точка на элементарном отрезке.

Пусть точка $c_i = x_{i-1} + \alpha * \Delta x$, где $\alpha \in \left[0;1\right]$ (в частности при $\alpha = 0.5$ $c_i$ - середина элементарного отрезка). Тогда, учитывая также, что $f(x) = \frac{1}{\sqrt{1-x^2}}$ и $\Delta x_i = \frac{\sqrt{3}}{n}$:
\begin{equation*}
\Delta S_i \approx\overline{\Delta S_i} =\frac{\sqrt{3}}{n} * \frac{1}{\sqrt{1-\left(-\frac{\sqrt{3}}{2} + \left(i-1 + \alpha\right)*\frac{\sqrt{3}}{n}\right)^2}} = \sqrt{\frac{3}{n^2-3\left(i-1+\alpha-\frac{n}{2}\right)^2}}
\end{equation*}
\item Запишем интегральную сумму:
\begin{equation}
S_n = \sum_{i=1}^{n} \Delta S_i \approx \sum_{i=1}^{n} \overline{\Delta S_i} =\sum_{i=1}^{n}\sqrt{\frac{3}{n^2-3\left(i-1+\alpha-\frac{n}{2}\right)^2}}
\end{equation},
где $\alpha \in \left[0;1\right]$ и показывает положение точки $c_i$ на элементарном отрезке.
\end{enumerate}
Исследуем значение $S_n$ при различных $n$ и положениях точки $c_i$ на элементарном отрезеке (то есть различных значениях $\alpha$):
\begin{center}
\begin{tabular}{l l l l}
\textbf{n} & \textbf{Левый край $\left(\alpha=0\right)$} & \textbf{Середина $\left(\alpha=0.5\right)$} & \textbf{Правый край$\left(\alpha=1\right)$}\\
\hline\\
3 & 2.361 & 1.992 & 2.361\\
\hline\\
6 & 2.176 & 2.058 & 2.176\\
\hline\\
10 & 2.126 & 2.079 & 2.126\\
\hline\\
50 & 2.096 & 2.094 & 2.096\\
\hline\\
100 & 2.095 & 2.094 & 2.095\\
\hline\\
\end{tabular}

\end{center}
Можно заметить, что с ростом $n$ разность между значениями $S_n$ при различных положениях точки $c_i$ уменьшается, что говорит об увеличивающейся точности приближения интегральной суммы.

Вычислим значение предела интегральных сумм, то есть вычислим интеграл от данной функции по отрезку:
\begin{equation*}
\int_{-\frac{\sqrt{3}}{2}}^{\frac{\sqrt{3}}{2}} \frac{1}{\sqrt{1-x^2}} dx
\end{equation*}
По теореме Ньютона-Лейбница:
\begin{equation*}
\int_{a}^{b} f(x) dx = \Phi(b) - \Phi(a)
\end{equation*},
где $\Phi(x)$ - первообразная $f(x)$.

По определению неопределённого интеграла:
\begin{equation*}
\Phi(x) + C = \int f(x)dx = \int \frac{1}{\sqrt{1-x^2}}dx = \arcsin{x} + C
\end{equation*}
Cледовательно:
\begin{equation*}
\int_{-\frac{\sqrt{3}}{2}}^{\frac{\sqrt{3}}{2}} \frac{1}{\sqrt{1-x^2}} dx = arcsin{\frac{\sqrt{3}}{2}} - arcsin{-\frac{\sqrt{3}}{2}} = \frac{\pi}{3} + \frac{\pi}{3} = \frac{2\pi}{3} \approx 2.094
\end{equation*}
Сравним значения интегральныъ сумм при различных $n$ с точным значением интеграла:


\newcommand{\sumPoint}[2]{
\draw [dotted, \scolor,  thick] (#2, #1) node[rectangle, fill, inner sep=1.5pt]{};
•}


\begin{figure}[H]
\begin{tikzpicture}
\begin{axis}[
	axis lines = left,
	xlabel = \(n\),
	ylabel = {$S_n$},
	ymin=0,
	xmin=0,
	xmax=20,
	grid=both,
    grid style={line width=.1pt, draw=gray!10},
    major grid style={line width=.2pt,draw=gray!50},
	ymax=5,
	xtick distance={2},
	]

\addplot[
	ultra thick,
    domain=0:20,
    samples=100,
    color=\pcolor,
    name path = f,
]
{2.094};
\addlegendentry{$\int f(x)$}
\addplot[only marks, thick, \scolor] table[y index= 0, x index = 1] {1.dat};
\addlegendentry{Левый и правый край}
\addplot[only marks, thick, tangerine] table[y index= 0, x index = 1] {2.dat};
\addlegendentry{Середина}

\end{axis}
\end{tikzpicture}

\caption{Сравнение интегральных сумм с точным значением интеграла}
\label{gr:integral}
\end{figure}
Анализируя рисунок \ref{gr:integral} можно заметить, что, во-первых, интегральные суммы для $c_i$ на левом и правом краю элементарного отрезка совпадают (что связано с симметричностью рассматриваемого отрезка и функции), а во-вторых - все последовательности сходятся к точному значению интеграла с ростом $n$.
\begin{finish}
По окончании исследования последовательности интегральных сумм можно заключить:
\begin{enumerate}
\item \textbf{Последовательности интегральных сумм сходятся} к точному значению интеграла с ростом $n$ вне зависимости от выбора точки $c_i$.
\item \textbf{Выбор точки $c_i$} существенно меняет значение интегральной суммы только при малых $n$. 
\end{enumerate}
\end{finish}
\section{2-e Задание}
\begin{task}
Найдите площадь плоской фигуры, ограниченной петлёй кривой $x= \frac{t^2}{1+t^2}, y=\frac{t(1-t^2)}{1+t^2}$.
\end{task}
\subsection{Решение}
Рассмотрим искомую фигуру на рисунке \ref{gr:curve}:
\begin{figure}[H]
\begin{tikzpicture}
\begin{axis}[
	axis lines = middle,
	xlabel = \(x\),
	ylabel = {\(y\)},
	ymin=-1,
	xmin=-0.5,
	xmax=1,
	ymax =1,
	grid=both,
    grid style={line width=.1pt, draw=gray!10},
    major grid style={line width=.2pt,draw=gray!50},
	xtick distance={0.5},
	]
	\addplot [domain=-2:2, samples=100, ultra thick, \pcolor, name path=f] ({(x^2)/(1+x^2)},{x*(1-x^2)/(1+x^2)});
	\path[name path = axis] (0,0) -- (0.5,0);

\addplot [
        thick,
        color=\scolor,
        fill=\scolor, 
        fill opacity=0.2
    ]
    fill between[
        of=f and axis,
        soft clip={domain=0:0.498},
    ];
\legend{$\left(\frac{t^2}{1+t^2}; \frac{t*(1-t^2}{1+t^2}\right)$, Искомая фигура}
\end{axis}
\end{tikzpicture}
\caption{Петля $f(x)$ с закрашенной искомой фигурой}
\label{gr:curve}
\end{figure}
Так как определённый интеграл Римана фактически равен площади под графиком функции, воспользуемся им для подсчёта площади данной фигуры.

Сначала определим промежуток параметра $t$ на котором расположеная петля кривой. На графике видно, что петля смыкается
при $y=0$, следовательно:
\begin{equation*}
\begin{aligned}
y &= \frac{t*(1-t^2)}{1+t^2}=0\\
t &= 0, t=\pm1\\
x(0) &= 0, x(-1) = x(1) = 0.5
\end{aligned}
\end{equation*}
Кривая дважды проходит через точку $\left(0.5, 0\right)$ при $t=\pm1$. При этом нижняя часть петли (ниже оси Ox) соответствует промежутку $t\in\left(-1, 0\right)$, а верхняя - $t\in\left(0, 1\right)$ (выше оси Ox). К тому же:
\begin{equation*}
y(-t) = -\frac{1*(1-t^2)}{1+t^2} = -y(t)
\end{equation*},
то есть $y(t)$ - нечётная функция, а значит - на симметричных промежутках ($\left[-1, 0\right]$ и $\left[0, 1\right]$) она принимает одинаковые по модулю, но разные по знаку значения. Значит, площадь фигуры под верхней веткой кривой равна площади над нижней веткой. Следовательно, площадь всей искомой фигуры можно посчитать как:
\begin{equation*}
S_{\Phi} = 2 * S_+
\end{equation*}, где $S_+$ - площадь под верхней частью петли.

Площадь под графиком функции $y(x)$ на отрезке $\left[a;b\right]$ равна:
\begin{equation*}
S = \int_a^b y(x)dx
\end{equation*}
При этом, если $y$ и $x$ выражены через параметр $t$, то:
\begin{equation*}
\int_a^b y(x)dx = \int_{t_1}^{t^2} y(t) dx(t) = \int_{t_1}^{t^2} y(t)x'(t) dt
\end{equation*}
Петля расположена выше оси OX при $t\in\left[0;1\right]$, следовательно:
\begin{equation*}
\begin{aligned}
S_{+} &= \int_{0}^{1} y(t)x'(t) dt\\
x'(t) &= \left(\frac{t^2}{1+t^2}\right)'=|\text{По св. произв. частн.}|=\frac{2*t*t^2 - 2 * t * (1+t^2)}{\left(1+t^2\right)^2} =\\
&= \frac{2*t}{\left(1+t^2\right)^2},\\
S_{+} &= \int_{0}^{1} \frac{t*(1-t^2)}{1+t^2} * \frac{2*t}{\left(1+t^2\right)^2} dt = \\
&= \int_{0}^{1} \frac{2*(t^2-t^4)}{\left(1+t^2\right)^3} dt=|\text{По св. линейности соб. инт.}|=\\
&= 2 * \left(\int_{0}^{1} \frac{t^2}{\left(1+t^2\right)^3} dt - \int_{0}^{1} \frac{t^4}{\left(1+t^2\right)^3} dt\right)\\
\end{aligned}
\end{equation*}
Рассмотрим каждый из суммируемых интегралов по отдельности:
\begin{equation*}
\begin{aligned}
\int_{0}^{1} \frac{t^2}{\left(1+t^2\right)^3} dt &= \int_{0}^{1} \frac{t^2 + 1 - 1}{\left(1+t^2\right)^3} dt =|\text{По св. лин. инт}| = \\
&= \int_{0}^{1} \frac{1}{\left(1+t^2\right)^2} dt - \int_{0}^{1} \frac{1}{\left(1+t^2\right)^3} dt
\end{aligned}
\end{equation*}

\begin{equation*}
\begin{aligned}
\int_{0}^{1} \frac{t^4}{\left(1+t^2\right)^3} dt &= \int_{0}^{1} \frac{t^4 + 2t^2 + 1 -2t^2 - 1}{\left(1+t^2\right)^3} dt = |\text{По св. лин. инт}|\\
&= \int_{0}^{1} \frac{1}{\left(1+t^2\right)} dt - \int_{0}^{1} \frac{1+2t^2}{\left(1+t^2\right)^3} dt =\\
&= \int_{0}^{1} \frac{1}{\left(1+t^2\right)} dt - \int_{0}^{1} \frac{2}{\left(1+t^2\right)^2} dt + \int_{0}^{1} \frac{1}{\left(1+t^2\right)^3} dt
\end{aligned}
\end{equation*}
Следовательно:
\begin{equation*}
\begin{aligned}
S_{+} = 2*\left( -\int_{0}^{1} \frac{1}{\left(1+t^2\right)} dt + \int_{0}^{1} \frac{3}{\left(1+t^2\right)^2} dt - \int_{0}^{1} \frac{2}{\left(1+t^2\right)^3} dt\right)\\
\end{aligned}
\end{equation*}
Решим каждый из интегралов, участвующих в сумме по отдельности:
\begin{equation*}
\begin{aligned}
\int_{0}^{1} \frac{1}{\left(1+t^2\right)} dt = |\text{Табл. инт}| = \arctan{t}|_0^1
\end{aligned}
\end{equation*}

\begin{equation*}
\begin{aligned}
\int_{0}^{1} \frac{3}{\left(1+t^2\right)^2} dt &=3 * \int_{0}^{1} \frac{1}{\left(1+t^2\right)^2} dt = |t = \tan{w}, w = \arctan{t}| =\\
&=3 * \int_{w_1}^{w_2} \frac{1}{\left(\tan{w}^2+1\right)^2} * \frac{1}{\cos{w}^2} dw = \\
&= 3 * \int_{w_1}^{w_2} \cos^4{w} * \frac{1}{\cos^2{w}} dw = |\text{По форм. пониж. степ.}| =\\
&= 3 * \int_{w_1}^{w_2} \frac{1+\cos{2w}}{2} dw = |\text{Внес. под дифф.}| =\\
&= \frac{3}{4} * \int_{w_1}^{w_2} 1+\cos{2w} d2w = \frac{3}{4}(2w +\sin{2w})|_{w_1}^{w_2} =\\
&= \left(\frac{3}{2}\arctan{t} + \frac{3}{4}\sin{2\arctan{t}}\right)|_0^1 =\\
&= |\sin{x}=\sqrt{\frac{\tan^2{x}}{1+\tan^2{x}}},\cos{x}=\sqrt{\frac{1}{1+\tan^2{x}}}, \sin{2x}=2*\sin{x}*\cos{x}| =\\
&= \left(\frac{3}{2}\arctan{t} + \frac{3}{4} * 2 *\sqrt{\frac{\tan^2{\arctan{t}}}{1+\tan^2{\arctan{t}}}}*\sqrt{\frac{1}{1+\tan^2{\arctan{t}}}}\right)|_0^1 =\\
&=\left(\frac{3}{2}\arctan{t} + \frac{3}{2}*\frac{t}{1+t^2}\right)|_0^1
\end{aligned}
\end{equation*}

\begin{equation*}
\begin{aligned}
\int_{0}^{1} \frac{2}{\left(1+t^2\right)^3} dt &= 2 * \int_{0}^{1} \frac{1}{\left(1+t^2\right)^3} dt = |t = \tan{w}, w = \arctan{t}| =\\
&=2 * \int_{w_1}^{w_2} \frac{1}{\left(\tan{w}^2+1\right)^3} * \frac{1}{\cos{w}^2} dw = \\
&= 2 * \int_{w_1}^{w_2} \cos^6{w} * \frac{1}{\cos^2{w}} dw = |\text{По форм. пониж. степ.}| =\\
&= 2 * \int_{w_1}^{w_2} \left(\frac{1+\cos{2w}}{2}\right)^2 dw=\\
&= \frac{1}{2} * \int_{w_1}^{w_2} 1 + 2\cos{2w} + cos^2{2w} dw=\\
&= |\text{По св. лин. и методу внес. под дифф.}| =\\
&= \frac{1}{2} \left(w|_{w_1}^{w_2} +  \int_{w_1}^{w_2} \cos{2w} d2w + \frac{1}{8} \int_{w_1}^{w_2} 1 + \cos{4w} d4w\right)  =\\
&= \frac{1}{2} \left(w +  \sin{2w} + \frac{1}{8} * 4w + \frac{1}{8} * \sin{4w}\right)|_{w_1}^{w_2} =\\
&=\left(\frac{1}{2} * \arctan{t} + \frac{t}{1+t^2} + \frac{1}{4} * \arctan{t} + \frac{1}{4} * \frac{t}{(1+t^2)^2} - \frac{1}{4} * \frac{t^3}{(1+t^2)^2}\right)|_0^1 =\\
&= \frac{1}{4} * \left(\frac{t*(3t^2+5)}{(t^2+1)^2} + 3*\arctan{t}\right)|_0^1
\end{aligned}
\end{equation*}
Возвращаемся к изначальной сумме:
\begin{equation*}
\begin{aligned}
S_{+} &= 2 * \left( -\arctan{t} + \frac{3}{2}\arctan{t} + \frac{3}{2}*\frac{t}{1+t^2} - \frac{1}{4} * \left(\frac{t*(3t^2+5)}{(t^2+1)^2} + 3*\arctan{t}\right)\right)|_0^1=\\
&= \frac{1}{2} * \left(\frac{3t^3+t}{(t^2+1)^2}-\arctan{t}\right)|_0^1
\end{aligned}
\end{equation*}
Тогда площадь фигуры равна:
\begin{equation}
S_\Phi = 2*S_{+} = \left(\frac{3t^3+t}{(t^2+1)^2}-\arctan{t}\right)|_0^1 = \frac{4}{4} - \frac{\pi}{4} - 0 = \frac{4-\pi}{4} \approx 0.215
\end{equation}
\subsection{Приблежение другими фигурами}
Приближённо найдём площадь данной фигуры при помощи треугольников:
\begin{figure}[H]
\begin{tikzpicture}
\begin{axis}[
	axis lines = middle,
	xlabel = \(x\),
	ylabel = {\(y\)},
	ymin=-1,
	xmin=-0.5,
	xmax=1,
	ymax =1,
	grid=both,
    grid style={line width=.1pt, draw=gray!10},
    major grid style={line width=.2pt,draw=gray!50},
	xtick distance={0.5},
	]
	\addplot [domain=-2:2, samples=100, ultra thick, \pcolor, name path=f] ({(x^2)/(1+x^2)},{x*(1-x^2)/(1+x^2)});
	\path[name path = axis] (0,0) -- (0.5,0);

\addplot [
        thick,
        color=\scolor,
        fill=\scolor, 
        fill opacity=0.2
    ]
    fill between[
        of=f and axis,
        soft clip={domain=0:0.498},
    ];
\draw[tangerine, fill] (0,0) node[anchor=north]{\large $A$}
  -- (0.2,-0.3) node[anchor=east]{\large $C$}
  -- (0.2,0.3) node[anchor=south]{\large $B$}
  -- cycle;
  \draw[purple, fill] (0.2,0.3) node[anchor=east]{\large $A'$}
  -- (0.2,-0.3) node[anchor=west]{\large $C'$}
  -- (0.5,0) node[anchor=south]{\large $B'$}
  -- cycle;
\legend{$\left(\frac{t^2}{1+t^2}; \frac{t*(1-t^2}{1+t^2}\right)$, Искомая фигура, s, s}
\end{axis}
\end{tikzpicture}
\caption{Приближение площади треугольниками}
\label{gr:curveByTrigs}
\end{figure}
Координаты вершин треугольника (подобраны исходя из графика):
\begin{equation}
\begin{aligned}
&A = (0,0), &B = (0.2,0.3), &C = (0.2,-0.3)\\
&A' =(0.2,0.3), &B' = (0.5,0),  &C' = (0.2,-0.3)
\end{aligned}
\end{equation}
Приближенная площадь:
\begin{equation*}
S_\Phi \approx S_{ABC} + S_{A'B'C'} = \frac{1}{2} * (0.2 * 0.6 + 0.3 * 0.6) = 0.15 
\end{equation*}
Приближенная площадь отличается от точной на: $\delta S = 0.215 - 0.15 = 0.065 \approx 30\% * S$. Следовательно, хоть предложенное приблежение и не очень точно, оно всё равно помогает оценить правдоподобность площади, вычисленной точно, (грубой прикидкой на рисунке \ref{gr:areaApprox} можно оценить, что от площади всей фигуры не покрытой осталось около $20-30\%$, что сходится с полученными результатами).
\section{3-е Задание}
\begin{task}
Найдите объём тела $T$, полученного вращением фигуры $\Phi$ вокруг оси Oy. Фигура $\Phi$ ограничена следующими кривыми:
\[
x = \sqrt{1-y^2}, y=\sqrt{\frac{3}{2}}x, y=0
\]
\end{task}
\subsection{Решение}
Изобразим на графике фигуру $\Phi$:
\begin{figure}[H]
\begin{tikzpicture}
\begin{axis}[
	axis lines = middle,
	xlabel = \(x\),
	ylabel = {\(y\)},
	ymin=-2,
	xmin=-2,
	xmax=2,
	ymax =2,
	grid=both,
    grid style={line width=.1pt, draw=gray!10},
    major grid style={line width=.2pt,draw=gray!50},
	xtick distance={0.5},
	]
	\addplot [domain=0:1, samples=100, ultra thick, \pcolor, name path=uparc] {(1-x^2)^0.5};
	\addplot [domain=0:1, samples=100, ultra thick, \pcolor, name path=dwnarc] {-(1-x^2)^0.5};
	\addplot [domain=-2:2, samples=100, ultra thick, purple, name path=stick] {(3/2)^0.5*x};
	\addplot [domain=-2:2, samples=100, ultra thick, imperial_red, name path=axis] {0};
\addplot [
        thick,
        color=\scolor,
        fill=\scolor, 
        fill opacity=0.2
    ]
    fill between[
        of=stick and axis,
        soft clip={domain=0:0.6325},
    ];
\addplot [
        thick,
        color=\scolor,
        fill=\scolor, 
        fill opacity=0.2
    ]
    fill between[
        of=uparc and axis,
        soft clip={domain=0.6325:1},
    ];
\node[circle, \scolor, fill, inner sep=2pt, thick, label = {180:{($\sqrt{\frac{2}{5}}$;$\sqrt{\frac{3}{5}}$)}}] at (0.63,0.77){};
\legend{$x=\sqrt{1-y^2}$,,$y=\sqrt{\frac{3}{2}}x$, $y=0$, фигура $\Phi$}
\end{axis}
\end{tikzpicture}
\caption{Кривые и ограничиваемая ими фигура $\Phi$}
\label{gr:area}
\end{figure}

Изобразим тело вращения $T$ (вращая фигуру $\Phi$ вокруг оси Oy):

\begin{figure}[H]
\begin{tikzpicture}
\pgfdeclarelayer{pre main}
\pgfsetlayers{pre main,main}

\begin{axis}[xlabel = $x$, ylabel = $z$, zlabel = $y$, grid= both,
xmax = 2,
xmin = -2,
ymax = 2,
ymin = -2,
zmax = 2,
zmin = -2,
z buffer=sort,
%colormap={mycol}{color=(aqua), color=(royal_blue)},
]
\addplot3[surf, opacity=0.4, point meta={abs(rawx+rawy+0.2*rawz)},
colormap={whiteblue}{color=(bloodred) color=(bloodred)},
samples=10,domain=0:1,y domain=0:2*pi, name path = sphere]
({x * cos(deg(y)},
{x * sin(deg(y)},
{0});

\addplot3[surf, opacity=0.4,point meta={abs(rawx+rawy+0.2*rawz)},
colormap={whiteblue}{color=(purple) color=(purple)},
samples=10,domain=0:0.775,y domain=0:2*pi, name path = stick]
({x *(2/3)^0.5 * cos(deg(y)},
{x *(2/3)^0.5 * sin(deg(y)},
{x});
\addplot3[surf, opacity=0.6, point meta={abs(rawx+rawy+0.2*rawz)},
colormap={whiteblue}{color=(peri) color=(peri)},
samples=20,domain=0:0.775,y domain=0:2*pi, name path = sphere]
({(1-x^2)^0.5 * cos(deg(y)},
{(1-x^2)^0.5 * sin(deg(y)},
{x});


\end{axis}
\end{tikzpicture}
\caption{Тело вращения $T$}
\label{gr:body}
\end{figure}
Объём тела вращения, образованного вращением фигуры, равен:
\begin{equation*}
V=\int_a^b S(x)dx =|\text{По св-ву инт.}|= \int_a^b S(y)dy
\end{equation*}
На рисунке \ref{gr:area} можно увидеть, что для тела $T$ функцию $S(y)$ можно задать как функцию для площади кольца ($S_{\text{кольцо}} = \pi * \left(R_{\text{внешний}}^2 - R_{\text{внутренний}}^2\right)$) в которой от свободной переменной $y$ зависят радиусы:
\begin{equation*}
\begin{aligned}
S(y) &= \left(f(y)_{\text{Внешний радиус}}^2 - f(y)_{\text{Внутренный радиус}}^2\right) * \pi=\\
&= \left(\left(\sqrt{1-y^2}\right)^2 - \left(\sqrt{\frac{2}{3}}y\right)^2\right)*\pi=\\
&=\left(1-\frac{5}{3}y^2\right)*\pi
\end{aligned}
\end{equation*},
где $f(y)_{\text{Внутренный радиус}}$ получена из прямой $y=\sqrt{\frac{3}{2}}x$, ограничивающей фигуру.

Для определения пределов интегрирования найдём точку пересечения ограничивающих кривых: $x=\sqrt{1-y^2}$ и $y=\sqrt{\frac{3}{2}}x$:
\begin{equation*}
\begin{aligned}
x =\sqrt{1-y^2} &= \sqrt{\frac{2}{3}}y\\
2.5y^2 &=1.5\\
y^2&=\frac{3}{5}\\
y&=\pm\sqrt{\frac{3}{5}}\\
\end{aligned}
\end{equation*}
Учитывая, что у искомой точки $x,y > 0$:
\begin{equation*}
\begin{aligned}
y &=\sqrt{\frac{3}{5}}\\
\end{aligned}
\end{equation*}
Вычислим объём тела:
\begin{equation*}
\begin{aligned}
V &= \int_{0}^{\sqrt{\frac{3}{5}}}  \left(1-\frac{5}{3}y^2\right)*\pi dy\\
&=\pi * \left(y-\frac{5}{3}*\frac{y^3}{3}\right)|_{0}^{\sqrt{\frac{3}{5}}}\\
&= \pi * \left(\sqrt{\frac{3}{5}}-\frac{5}{3}* \frac{3}{5} *\sqrt{\frac{3}{5}} * \frac{1}{3}\right) - 0=\\
&=\frac{2\sqrt{3}}{3\sqrt{5}}*\pi \approx 1.622
\end{aligned}
\end{equation*}
\subsection{Приблежение другими телами}
Аппроксимируем объём тела вращения $T$ кольцами:
\begin{figure}[H]
\begin{tikzpicture}
\begin{axis}[
	axis lines = middle,
	xlabel = \(x\),
	ylabel = {\(y\)},
	ymin=-2,
	xmin=-2,
	xmax=2,
	ymax =2,
	grid=both,
    grid style={line width=.1pt, draw=gray!10},
    major grid style={line width=.2pt,draw=gray!50},
	xtick distance={0.5},
	]
	\addplot [domain=0:1, samples=100, ultra thick, \pcolor, name path=uparc] {(1-x^2)^0.5};
	\addplot [domain=0:1, samples=100, ultra thick, \pcolor, name path=dwnarc] {-(1-x^2)^0.5};
	\addplot [domain=-2:2, samples=100, ultra thick, purple, name path=stick] {(3/2)^0.5*x};
	\addplot [domain=-2:2, samples=100, ultra thick, imperial_red, name path=axis] {0};
\addplot [
        thick,
        color=\scolor,
        fill=\scolor, 
        fill opacity=0.2
    ]
    fill between[
        of=stick and axis,
        soft clip={domain=0:0.6325},
    ];
\addplot [
        thick,
        color=\scolor,
        fill=\scolor, 
        fill opacity=0.2
    ]
    fill between[
        of=uparc and axis,
        soft clip={domain=0.6325:1},
    ];

\legend{$x=\sqrt{1-y^2}$,,$y=\sqrt{\frac{3}{2}}x$, $y=0$, фигура $\Phi$}

\draw[purple, fill] (0.2,0) node[circle, black, fill, inner sep=1.3pt, thick, label = {[font=\tiny,text=black]180:{($0.2$;$0$)}}]{}
  -- (1,0) node[circle, black, fill, inner sep=1.3pt, thick, label = {[font=\tiny,text=black]0:{($1$;$0$)}}]{}
  -- (1,0.25) node[circle, black, fill, inner sep=1.3pt, thick, label = {[font=\tiny,text=black]0:{($0.2$;$0.25$)}}]{}
  -- (0.2,0.25) node[circle, black, fill, inner sep=1.3pt, thick, label = {[font=\tiny,text=black]180:{($1$;$0.25$)}}]{}
  -- cycle;
\draw[tangerine, fill] (0.4,0.25) node[circle, black, fill, inner sep=1.3pt, thick, label = {[font=\tiny,text=black]:{($0.4$;$0.25$)}}]{}
  -- (0.87,0.25) node[circle, black, fill, inner sep=1.3pt, thick, label = {[font=\tiny,text=black]25:{($0.87$;$0.25$)}}]{}
  -- (0.87,0.49) node[circle, black, fill, inner sep=1.3pt, thick, label = {[font=\tiny,text=black]0:{($0.87$;$0.49$)}}]{}
  -- (0.4,0.49) node[ circle, black, fill, inner sep=1.3pt, thick,label = {[font=\tiny,text=black]180:{($0.4$;$0.49$)}}]{}
  -- cycle;
  \draw[purple, fill] (0.54,0.49) node[circle, black, fill, inner sep=1.3pt, thick, label = {[font=\tiny,text=black]155:{($0.54$;$0.49$)}}]{}
  -- (0.75,0.49) node[circle, black, fill, inner sep=1.3pt, thick, label = {[font=\tiny,text=black]25:{($0.75$;$0.49$)}}]{}
  -- (0.75,0.66) node[circle, black, fill, inner sep=1.3pt, thick, label = {[font=\tiny,text=black]25:{($0.75$;$0.66$)}}]{}
  -- (0.54,0.66) node[circle, black, fill, inner sep=1.3pt, thick, label = {[font=\tiny,text=black]155:{($0.54$;$0.66$)}}]{}
  -- cycle;
\end{axis}
\end{tikzpicture}
\caption{Приближение фигуры $T$ кольцами - сечения. }
\label{gr:areaApprox}
\end{figure}
Для каждого кольца определим $R_{\text{внутренний}}$, $R_{\text{внешний}}$ b высоту $H$:
\begin{equation*}
\begin{aligned}
&R_{i1} = 0.2, &R_{o1} = 1, &H_1 = 0.25\\
&R_{i2} = 0.4, &R_{o2} = 0.87, &H_2 = 0.24\\
&R_{i3} = 0.54, &R_{o3} = 0.75 &H_3 = 0.17\\
\end{aligned}
\end{equation*}, где $R_i$ - внутренний радиус, $R_o$ - внешний радиус, $H$ - высота кольца.

Для вычисления объёма кольца воспользуемся формулой:
\begin{equation*}
V_{\text{кольцо}} = \pi * \left(R_{o}^2 - R_{i}^2\right) * H
\end{equation*}
Найдём объём каждого из колец:
\begin{equation*}
\begin{aligned}
V_1 &= \pi * \left(1^2 - 0.2^2\right) * 0.25&= 0.24\pi\\
V_2 &= \pi * \left(0.87^2 - 0.4^2\right) * 0.24&= 0.14\pi \\
V_3 &= \pi * \left(0.75^2 - 0.54^2\right) * 0.17&= 0.08\pi\\
\end{aligned}
\end{equation*}, таким образом общий приближенный объём составляет:
\begin{equation*}
V_T \approx V_1 + V_2 + V_3 = 0.46\pi \approx 1.445\\
\end{equation*}
Приближенный объём отличается от точного на: $\delta V = 0.1.622 - 1.445 = 0.117 \approx 11\% * V$. Следовательно, найденная аппроксимация свидетельствует о правдободобности точного объёма тела $T$, найденного с использованием интеграла.
\newpage
\section{3-е Задание}
\begin{figure}[H]
\centering

\pgfdeclarelayer{bg}
\pgfsetlayers{bg,main}
\begin{tikzpicture}[baseline=(current bounding box.north), scale = 3]

\draw[name path = cauldron,  thick, fill = peri, draw = black, line width = 1pt] (-1.5,0) -- (1.5,0) arc(0:-180:1.5) --cycle;

\draw[name path = newLevel, draw = black, line width = 1pt] (-1.42,-0.5) -- (1.42,-0.5);
\begin{pgfonlayer}{main}
\fill [orange!50,
          intersection segments={
            of=cauldron and newLevel,
            sequence={L2--R2}
          }];
\end{pgfonlayer}

\draw[-stealth, ultra thick] (0,-0.5) node[circle, fill, black, inner sep=1.5pt ]{}  -- (0,0.2) node[label = {$\overrightarrow{F_{\text{в.}}}$}]{};
\draw[-stealth, ultra thick] (0,-0.5) node[circle, fill, black, inner sep=1.5pt ]{}  -- (0,-1.2) node[label = {180:$\overrightarrow{P_{\text{c}}}$}]{};

\draw[|-|, dashed, ultra thick] (-1,-0.5) -- (-1,0);
\node[label ={$\Delta h$}] at (-0.85, -0.4) {};

\draw[|-, dashed, ultra thick] (0,-1.5) -- (1.5,-1.5);
\node[label ={270:$R$}] at (0.75, -1.5) {};
\draw[|-, dashed, ultra thick] (1.5,0) -- (1.5,-1.5);
\node[label ={360:$R$}] at (1.5, -0.75) {};

%
%%Labels for the vertices are typeset.

\end{tikzpicture}
\caption{К условию задачи - котёл с маслом. }
\label{gr:physics}
\end{figure}
\section{Выводы}
Как прекрасна РГР в тёплый весенний денёк! Тимур Рамеев, гад, где твои задания?!
- Тимур просил передать, что он очень болен - в связи с этим, его часть работы... ну...
\newpage
\section{Оценочный лист}
\begin{center}
\large
\begin{tabular}{|l|l|l|l|r|}
\hline
Хороших & Дмитрий & P3117 & 1.5 & 100\%\\
\hline
Рамеев & Тимур & P3118 & 1.5  & ?\%\\
\hline
\end{tabular}
\end{center}

\end{document}
